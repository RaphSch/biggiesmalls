\title{On dwarf elephants and giant rabbits}
\author{
        Rapha\"{e}l Scherrer \\
                University of Groningen\\
}
\date{\today}

\documentclass[12pt]{article}

\begin{document}
\maketitle

Welcome to the biggiesmalls project! Biggiesmalls is a model of body size evolution on islands.

\section{Background}

The island rule is a common pattern in animals, especially in mammals (Lomolino 1985, 2005). It refers to the tendency of large species to become smaller (i.e. dwarfism), and of small species to become larger (i.e. gigantism) upon colonization of an island. Two extreme (extinct) cases of this phenomenon are the lagomorph \textit{Nuralagus rex} from Minorca and the dwarf elephant \textit{Palaeoloxodon fosteri} from Sicily. Even though these examples are now extinct, the pattern holds for many extant species (Lomolino 2005, Lomolino et al. 2012).\\

Several explanations have been proposed to explain this phenomenon, among which e.g. the ecological release hypothesis, the immigration selection hypothesis or the resource limitation hypothesis.\\

The ecological release hypothesis states that because of competitive interactions between the numerous species present on the mainland, species have been driven away from each other in body size through character displacement. In this context, islands represent oases free of competition because they carry less species. This hypothesis provides an explanation for both dwarfism of large mammals and gigantism of small mammals. The prediction of this theory is that dwarfism and gigantism should be more pronounced on islands with less competing species.\\

The immigration selection hypothesis suggests that, in small animals, larger individuals are positively selected during the colonization process because they can survive better long journeys at sea (e.g. because they have larger fat reserves). This hypothesis offers an explanation to gigantism of small mammals, but not to dwarfism of large mammals. It predicts that more isolated islands should show more pronounced gigantism.\\

The resource limitation hypothesis states that large mammals dwarf on islands because resources are more scarce than on the mainland, therefore large bodied animal populations cannot be sustained. This offers an explanation for dwarfism of large mammals, but not to gigantism of small mammals. It predicts that dwarfism should be more pronounced on smaller islands (assuming resources are rarer on smaller islands).\\

In 2012, Lomolino et al. investigated potential biotic and abiotic factors that explain best the patterns of gigantism and dwarfism on islands worldwide. They found strong support for several factors, including the ecological release hypothesis (body size changes are more pronounced on islands with less competitors) and the immigration selection hypothesis (small species grow bigger on more isolated islands). However, they found no support for the resource limitation hypothesis, i.e. they did not see more pronounced dwarfism on smaller islands (independently of the presence of competitors). What is more, they actually found more pronounced gigantism in small species on smaller islands, a result that is at odds with what the resource limitation hypothesis predicts. Indeed, resource limitation should not affect small species and therefore dwarfism should not be affected by resource quantity.\\

Here, we propose a model to explain this pattern. There may be several reasons why, independently of other factors such as the number of competitors on the island, dwarfism is not affected by island area but gigantism is more pronounced on smaller islands.

\section{Hypotheses}

We propose that increased gigantism in small species on smaller islands is driven by asymmetric competition. It is easily conceivable that within a species, larger individuals are favored when acquiring food because they usually win competitive interactions with smaller individuals. On smaller islands, resource is more scarce and this advantage of being bigger may become critical and induce a strong selection for larger body sizes.\\

However, this should also apply to large species i.e. large species should also increase in body size on smaller islands, which is contrary to the observed pattern. One possible explanation is genetic drift. If larger animals have lower carrying capacities (i.e. the island can support less individuals), then population size of large animals may become too small on small islands for selection to really have an effect. Instead, genetic drift is expected to have a strong effect on the evolution of body size on smaller islands, which may explain the absence of response to selection and the absence of relationship with island area.\\

One important point is that we assume that resource limitation has two antagonistic effects. First, resource limitation induces selection for larger body sizes because of asymmetric competition, which is an aspect that has been overlooked in the literature. Second, larger body sizes have lower carrying capacities, which reflects the classically expected consequence of scarce resource i.e. that populations of too large individuals cannot be sustained if the resource is too scarce.\\

Another possible explanation for the absence of relationship between dwarfism of large mammals and island area is generation time. Larger mammals typically have longer generation times, therefore mutations accumulate more slowly and evolution (response to selection) is expected to take more time than in small species. We also know that islands are usually short-lived and therefore the islands we see nowadays are relatively young. It may be that resource limitation induces selection for increased dwarfism in large mammals, but that the response to selection takes too long to be visible yet, and that islands disappear before animals have responded to selection.\\



\section{The model}

Biggiesmalls is an individual-based simulation model, meaning it simulates each individual animal in a population. More precisely, it is a birth-death process, meaning that only two events can happen to an individual with body size $x_i$: it can either give birth (for simplicity, here we assume asexual repoduction by fission i.e. production of one offspring) at rate $B(x_i)$, or die at rate $D(x_i)$.\\

The birth rate of an individual is:

\[
B(x_i) = R(x_i) \Bigg(1 - \frac{\sum_{j = 1, j \neq i}^{n} C(x_i - x_j)}{K(x_i)} \Bigg)
\]

where $R(x_i)$ is the fecundity (in number of offpsring per unit time, which also represents generation time), and depends on body size according to the allometric relationship (Blueweiss et al. 1978):

\[
R(x_i) = 0.025 x_i^{-0.25}
\]

The numerator of the fraction between brackets represents the effect of asymmetric competition, where $C(x_i - x_j)$ is the competition kernel i.e. the cost for an individual $i$ to compete with an individual $j$. This cost is summed over all individuals in the population to yield a net cost of competition. $C(x_i - x_j)$ is a sigmoid function whose slope at zero is controlled by $\alpha$, the intensity of asymmetric competition. This means that if $j$ is larger than $i$, $C(x_i - x_j)$ will be positive, meaning that $i$ will experience a positive cost (i.e. $i$ will "loose" some food in the competition). But if $i$ is larger than $j$, the cost will be negative and $i$ will "gain" something from the competition. The outcome of competition between two individuals can never be higher than $1$ or lower than $-1$. Imagine that each individual has one unit of food before competition: upon competition, an individual cannot steal more food than the competitor has, nor can it loose more food than it itself possesses.\\

\[
C(x_i - x_j) = 1 - \frac{2}{1 + e^{-\alpha (x_i - x_j)}}
\]

The carrying capacity, $K(x_i)$ reflects the number of individuals of size $x_i$ that the island can support. Carrying capacity is an allometric function of body size:

\[
K(x_i) = K_0 x_i ^\kappa
\]

where $K_0$ is the basal carrying capacity (for individuals of size $0$) and $\kappa$ is a scaling factor describing how the carrying capacity decreases with body size.\\

As I said before, individuals also die at rate $D(x_i)$:

\[
D(x_i) = \frac{1}{l(x_i)}
\]

where $l(x_i)$ is the longevity, which is an allometric function of body size (Blueweiss et al. 1978):

\[
l(x_i) = 630 x_i^{0.17}
\]

This equation reflects the fact that larger individuals have longer lifespans.

\section{On to the simulation}

You will find the R script for the simulation on my GitHub repository. Go to GitHub, and look for RapSch. You will find a folder there with the code, in a file called biggiesmalls.R. Don't confuse it with biggiesmalls.txt, which is the C++ version of the program!\\

The code is organized in the following parts: (1) the parameters, which you can change, (2) some accessory functions that are used to calculate the birth and death rates, (3) the main function, which contains all the information to run the simulation and (4) a command to actually launch the simulation. Have a look at the code and try to understand what every line is doing. If you have questions, don't hesitate to ask on Whatsapp. To run the code on your computer, just open the script in RStudio, set the parameters as you wish, then Ctrl+A to select all the text and Ctrl+Enter to run it all. At each birth or death, the simulation time and population size will be printed to the command prompt.\\

Your exercise for Monday: determining what are the parameters you could play with to test the hypotheses we propose for the observed pattern of no relationship of dwarfism with island size and increased gigantism on smaller islands. Then, we'll discuss your propositions and set a plan for what simulations to do. You will then play a bit with the simulation on your laptops, but in a few days you'll be able to launch them on the cluster (Rampal has requested access for you). I will show you next week how to launch stuff on the cluster. Good luck!\\

\end{document}